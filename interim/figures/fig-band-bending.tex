% figures/fig-band-bending.tex
\begin{figure}[H]
  \centering
  \vspace{-0.75cm} % lift drawing to reduce apparent whitespace
  \begin{tikzpicture}[x=1.2cm,y=0.7cm,>=Stealth,font=\footnotesize]

    % Axes labels
    \node[anchor=north] at (2.2,0) {AlGaAs};
    \node[anchor=north] at (5.7,0) {GaAs};

    % Fermi level reference
    \draw[red, dashed] (0.2,3) -- (6.8,3);
    \node[anchor=east] at (0.1,3) {$E_F$};

    % Conduction band (bend into a narrow well then flatten in GaAs)
    \draw
      (0.2,3.3) -- (3.5,3.3)
      .. controls (4.5,3.3) and (4.5,6) .. (4.5,2.5)
      .. controls (5,3.8) and (5.3,3.8) .. (5.4,3.8)
      -- (6.6,3.8);

    % Valence band (S-bend at the interface then flat)
    \draw
      (0.2,0) -- (3.8,0)
      .. controls (4.5,0) and (4.5,0.6) .. (4.5,1)
      .. controls (4.5,1.5) and (5,1.5) .. (5.6,1.5)
      -- (6.6,1.5);

    % Energy labels on left
    \node[anchor=east] at (0.1,3.5) {$E_C$};
    \node[anchor=east] at (0.1,0) {$E_V$};

    % Highlight a small triangular 2DEG region in the well
    \filldraw[blue!70] (4.5,2.5) -- (4.5,3) -- (4.7,3) -- cycle;

    % 2DEG indication at the well minimum
    \draw[->,blue,thick] (6.5,2.5) -- (4.65,2.7);
    \node[blue,anchor=south west] at (6.4,2.25) {2DEG};

  \end{tikzpicture}
  \vspace{5mm}
  \caption{Band Structure Diagram of the AlGaAs/GaAs interface demonstrating the resulting Two-Dimensional Electron Gas (2DEG) formed in a Potential Well.}
  \label{fig:band_bending}
\end{figure}
