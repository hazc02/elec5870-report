\chapter{Introduction and Technical Background}

\section{Motivation and Context}

High-electron-mobility transistors (HEMTs) and their pseudomorphic variants 
(pHEMTs) are central to many modern high-frequency and high-performance 
electronic systems. Their exceptional electron mobility and low parasitic 
capacitance make them indispensable in microwave amplifiers, radar receivers, 
satellite communications, and other applications where high gain and low noise 
are required. These advantages arise from the formation of a two-dimensional 
electron gas (2DEG) within a III–V heterostructure, where the material pairing 
can vary depending on the application. Systems based on gallium nitride (GaN), 
indium phosphide (InP), and gallium arsenide (GaAs) are among the most common. 
In this project, GaAs has been selected due to its maturity, availability 
within the Leeds cleanroom, and well-characterised processing behaviour.

In contrast to conventional silicon metal–oxide–semiconductor field-effect 
transistors (MOSFETs), which rely on carrier inversion within a doped substrate, 
HEMTs achieve conduction through modulation of a 2DEG confined at a 
heterojunction interface. This separation between the carrier channel and dopant 
impurities yields far higher mobility; as a result, HEMTs can operate in the 
tens to hundreds of gigahertz range. The heterostructure design, however, makes 
control over channel depletion and threshold voltage more complex than in 
conventional transistors, as these parameters depend sensitively on epitaxial 
structure, surface condition, and gate geometry. This complexity has limited the 
widespread use of HEMTs in digital applications, where consistent and 
well-defined switching characteristics are essential.

Although HEMTs dominate within analogue and radio-frequency domains, their use 
in digital logic has remained limited. Most GaAs-based field-effect transistors 
operate in depletion mode, meaning that they conduct current at zero gate bias 
and must be driven negatively to turn off. Logic circuits constructed solely 
from depletion-mode transistors suffer from high static power dissipation, 
incomplete switching, and restricted noise margins. Enhancement-mode devices, 
in contrast, remain non-conductive at zero bias and switch on only when a 
positive gate voltage is applied; these characteristics are desirable for logic 
implementation, providing defined logic levels and low static power.

To realise efficient GaAs logic, both device types are required. Complementary 
operation between depletion- and enhancement-mode transistors underpins logic 
families such as direct-coupled FET logic (DCFL) and enhancement/depletion-load 
logic. The depletion device provides a passive load, while the enhancement 
device acts as the active pull-down element. Together they enable rail-to-rail 
voltage swings and reduced power consumption. Achieving reliable 
enhancement-mode behaviour within standard GaAs fabrication processes, however, 
remains a significant challenge, particularly in facilities without etch-stop 
layers or advanced epitaxial control.

\section{Rationale for Project Direction}

During the early phase of this project, sixteen depletion-mode pHEMTs were 
fabricated and characterised in the Leeds Nanotechnology Cleanroom. The devices 
exhibited threshold voltages around \(-0.9~\mathrm{V}\) and peak 
transconductance near \(3~\mathrm{mS}\), confirming that the existing process 
produces high-quality depletion-mode transistors. While these results 
established a strong foundation, they also revealed a key limitation: the 
absence of enhancement-mode devices prevents the exploration of complementary 
logic behaviour.

This limitation defines the central direction of the project. Achieving 
enhancement-mode operation would enable the construction of logic families that 
combine depletion- and enhancement-mode devices on the same substrate, improving 
switching efficiency and enabling more complex circuit design. To pursue this 
goal, the project focuses on whether controlled gate-recess etching can be used 
to shift the threshold voltage of GaAs pHEMTs into the positive region, without 
altering the wafer’s epitaxial structure or relying on etch-stop layers.

The approach is motivated by both practical and academic considerations. From a 
practical standpoint, demonstrating enhancement-mode behaviour using only the 
processes available within the Leeds cleanroom would represent a significant 
extension of current fabrication capability. From an academic perspective, it 
offers an opportunity to investigate how nanoscale recess depth influences 
electrostatic control, surface morphology, and threshold stability. Together, 
these aims provide a focused route towards developing complementary GaAs logic 
within an accessible and resource-constrained environment.

\section{pHEMT Device Physics Background}

A pseudomorphic HEMT consists of a GaAs channel beneath an AlGaAs donor layer. 
Electrons transfer from the donor layer into the GaAs, forming a two-dimensional 
electron gas at the heterojunction. The 2DEG provides a highly conductive channel 
whose carrier concentration is modulated by the Schottky gate potential. The 
threshold voltage depends primarily on the gate–channel separation, barrier 
composition, and surface charge density.

In depletion-mode devices, the 2DEG exists at zero gate bias, and a negative 
voltage is required to deplete the carriers and switch the device off. In 
enhancement-mode devices, the 2DEG is fully depleted when the gate voltage is 
zero, and a positive voltage is required to induce conduction. One of the most 
effective methods of shifting the threshold voltage towards positive values is 
to recess the gate. By carefully etching away a controlled thickness of the 
AlGaAs barrier, the gate potential gains stronger electrostatic influence over 
the channel, resulting in a shallower depletion region and a positive shift in 
threshold voltage.

Excessive etching increases leakage and roughens the surface, whereas 
insufficient etching yields little or no threshold shift. The process therefore 
demands precise control at nanometre-scale depths. A schematic cross-section of 
a pHEMT structure will be included to illustrate the formation of the 2DEG and 
the influence of the recessed gate region on the electrostatic potential profile.

\section{Project Aim and Significance}

This project aims to develop and experimentally validate a controllable 
gate-recess process capable of inducing enhancement-mode behaviour in GaAs-based 
pHEMTs using only the equipment available within the Leeds Nanotechnology 
Cleanroom. The work involves calibrating etch rates, fabricating transistors with 
systematically varied recess depths, and evaluating the resulting electrical 
characteristics. From this, an empirical relationship between recess depth and 
threshold voltage will be established.

Beyond the immediate fabrication results, the broader significance lies in 
demonstrating enhancement-mode operation without dedicated epitaxial or 
etch-stop structures. Achieving this would show that complementary GaAs logic 
could be realised using existing, low-cost infrastructure. Such a process would 
also offer energy-efficiency benefits in high-speed electronics, supporting 
sustainable design principles by reducing static power dissipation in compound 
semiconductor logic.

The findings are expected to contribute to a deeper understanding of 
threshold-voltage engineering, surface passivation, and process sensitivity in 
III–V semiconductor technology, while providing a foundation for future logic 
circuit demonstrations, such as DCFL inverters and NAND gates, constructed from 
both depletion- and enhancement-mode devices.