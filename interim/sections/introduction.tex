\chapter{Introduction}

\section{Context and Motivation}

Modern electronic and communication systems impose increasingly demanding
requirements on the devices that form their signal paths. Higher carrier
frequencies, lower permissible noise levels, stricter linearity constraints
and improved power efficiency are now expected. At the device level, these
system requirements manifest as targets for cut-off frequency, noise figure,
gain and power handling that conventional silicon technologies cannot always
achieve.

High-electron-mobility transistors meet many of these system requirements 
through the use of a carefully engineered heterojunction, which is the 
interface between two semiconductor materials with different band structures.
This band discontinuity causes electrons to collect in a very thin region 
at the heterojunction, creating a highly mobile sheet of charge carriers known as a 
two-dimensional electron gas (2DEG). The key advantage of this arrangement
is that the 2DEG forms without the need to place donor atoms directly in the 
conduction path. In contrast to MESFETs, where electrons move through a doped 
channel, the absence of ionised impurities in a HEMT channel greatly reduces 
Coulomb scattering \cite{ali1991}. As a result, electrons can move more freely,
giving the device high mobility, high transconductance and predictable behaviour
at high frequencies. These characteristics allow HEMTs to operate reliably 
across a broad frequency range, typically from around 1 GHz to beyond 100 GHz \cite{liu1999},
making them well suited to low-noise amplifiers, receiver front ends and other high-frequency subsystems.

Within this context, it is necessary to understand how the structure and
processing of III–V HEMTs influence their electrical behaviour. Practical
device performance is shaped not only by the underlying material system,
but also by the fabrication steps used to define the channel, contacts and
gate. Engaging directly with these processes provides a clearer view of how
device characteristics emerge from both design and implementation. This
interim report therefore introduces the key principles that underpin HEMT
operation and presents the fabrication and analysis of GaAs-based
pseudomorphic III–V HEMT devices produced during the first phase of the
project.


\section{Importance of III--V HEMTs}

III--V compound semiconductors play a central role in enabling the
performance characteristics associated with HEMT devices. Materials such as
GaAs, InP and GaN possess band structures and carrier transport properties
that support high-frequency operation, low noise and, in some cases, high
power density. \cite{liu1999} Their relatively low effective electron mass and favourable
mobility characteristics provide a foundation for transistor behaviour that
is difficult to achieve with silicon CMOS or SiGe bipolar processes over the
same frequency range. \cite{}

The heterostructure engineering available within III--V systems allows
precise control over carrier confinement, channel composition and interface
quality. By separating charge carriers from their parent dopants, these
structures sustain mobility values that directly translate into high
transconductance and improved gain at microwave frequencies. As a result,
III--V HEMTs continue to serve key functions in low-noise receivers,
high-frequency amplifiers and specialist instrumentation.

\section{Current Landscape and Relevance}

Although silicon RF technologies have progressed significantly, particularly
in CMOS and SiGe BiCMOS platforms, III--V HEMTs remain important in domains
where the highest noise performance, highest frequency operation or highest
power density are required. Their continued use in satellite communication
payloads, microwave backhaul links, scientific instrumentation and radar
receivers reflects the practical advantages of III--V material systems in
these regimes.

Contemporary research and industry practice continue to refine III--V device
structures, epitaxial growth methods and fabrication processes to improve
linearity, noise behaviour and frequency response. Even where silicon-based
technologies dominate large-volume commercial applications, III--V HEMTs
retain a critical role in performance-driven systems that cannot rely solely
on silicon.

\section{Scope and Structure of the Report}

This interim report introduces the fundamental operating principles of
III--V HEMT devices and describes the fabrication and characterisation work
completed during the first phase of the project. The report begins with the
physical structure and behaviour of pseudomorphic HEMTs, followed by a
summary of the cleanroom fabrication processes used to produce the devices
analysed here. Their electrical characteristics are then examined through
transfer behaviour, output curves, transconductance and threshold voltage
extraction.

The final sections provide a reflection on project management and skills
development, before outlining the planned technical work for the second
semester.
