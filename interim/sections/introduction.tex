\chapter{Introduction}

\section{Context and Motivation}

Modern communication, sensing and microwave systems impose demanding 
requirements on the semiconductor devices that form their front-end 
signal paths. Higher operating frequencies, lower permissible noise 
figures and stricter linearity constraints continue to push transistor 
technologies to their limits. Although silicon-based radio-frequency 
platforms have made notable progress in recent years, they do not always 
provide the carrier mobility, noise performance or high-frequency gain 
required in performance-critical microwave and millimetre-wave 
applications.

III–V high-electron-mobility transistors (HEMTs) have therefore retained 
a central role in many advanced electronic systems. Their advantageous 
material properties, particularly the high electron mobility achievable 
in modulation-doped heterostructures, support predictable, low-noise 
operation over broad frequency ranges \cite{ali1991}. As a result, HEMTs 
continue to underpin satellite communication payloads, radar receivers, 
millimetre-wave backhaul links and a range of scientific instruments 
where stringent performance metrics exceed the capabilities of silicon 
technologies alone.

\section{Relevance of GaAs/AlGaAs HEMTs to This Project}

In recent years, gallium nitride (GaN) HEMTs 
have become the dominant technology for high-power and high-linearity 
applications. Their wide bandgap, high breakdown field and favourable 
impedance characteristics enable power handling and efficiency that 
surpass those of GaAs-based devices. Despite this shift, 
GaAs/AlGaAs HEMTs remain one of the most mature and extensively studied 
III–V transistor platforms.

For the purposes of this project, GaAs offers several practical and 
technical advantages. The Leeds Nanotechnology Cleanroom maintains 
well-established processes for GaAs/AlGaAs heterostructures, 
and suitable wafers are readily available with highly reproducible 
epitaxial quality. This maturity and process stability make GaAs an ideal 
material system for exploring how heterostructure design and fabrication 
choices influence final device behaviour. The platform is particularly 
well suited to developing practical skills in device definition while 
enabling detailed investigation of threshold behaviour, transconductance, 
leakage mechanisms and other key electrical characteristics.

\section{Project Motivation and Aims}

The primary aim of this work is to fabricate and characterise 
depletion-mode GaAs/AlGaAs HEMTs in order to examine the relationship 
between material structure, process conditions and device performance. 
By producing baseline devices using the cleanroom's established process 
chain, the project seeks to understand how variations in lithography, 
etching accuracy, ohmic contact formation and gate metallisation affect 
the electrical characteristics that emerge from the heterostructure.

A broader motivation is to develop a rigorous understanding of the 
interplay between epitaxial design and practical fabrication. This 
includes assessing the reproducibility of each processing stage, the 
sensitivity of device behaviour to small deviations in patterning and 
etch depth, and the role of interface quality in shaping threshold 
voltage, transconductance and leakage performance. Establishing this 
foundation provides both the technical context and the practical 
experience needed to support the more advanced device development 
planned for the second semester.
