\chapter{Introduction}

Modern communication, sensing and microwave systems impose increasingly 
demanding requirements on the semiconductor devices that form their 
front-end signal paths. Higher operating frequencies, tighter noise 
constraints and strict linearity targets continue to push transistor 
technologies beyond the capabilities of many silicon-based platforms. 
Although silicon processes have made notable progress in recent years, 
they do not always provide the carrier mobility, noise performance or 
high-frequency gain required in performance-critical microwave and 
millimetre-wave applications.

III–V High-Electron-Mobility Transistors (HEMTs) play a central role in 
addressing these challenges. Their favourable material properties, 
particularly the high electron mobility achievable within 
modulation-doped heterostructures, enable predictable, low-noise 
operation across wide frequency ranges \cite{ali1991}. As a result, 
HEMTs remain fundamental to satellite communication payloads, radar 
receivers, millimetre-wave backhaul links and a range of scientific and 
instrumentation systems where stringent performance metrics cannot be 
met by silicon technologies alone.

Research activity within the field has increasingly shifted toward 
gallium nitride (GaN) HEMTs \todobib, driven by their wide bandgap, high 
breakdown field and strong power-handling capabilities. These 
characteristics make GaN the preferred option for many high-power and 
high-linearity applications \todobib. Despite this, Gallium Arsenide GaAs 
HEMTs remain one of the most mature, well-understood and experimentally 
accessible III–V transistor technologies. Their stability, extensive documentation and 
long industrial history continue to make them an invaluable platform for 
studying the fundamental relationships between heterostructure design, 
device architecture and electrical behaviour.

For the purposes of this project,  GaAs presents clear practical and 
technical advantages. The Leeds Nanotechnology Cleanroom maintains 
established, reliable processes for GaAs/AlGaAs heterostructures, and 
suitable wafers are readily available with reproducible epitaxial 
quality \cite{leedscleanroom}. This provides a controlled environment in 
which HEMT structures can be fabricated and analysed systematically. The 
work undertaken here aims to investigate how variations in device 
structure influence key electrical characteristics, including threshold 
voltage, transconductance, leakage behaviour and the general conduction 
profile of the device. By examining how different architectural choices 
and design modifications affect these metrics, the project seeks to 
develop a coherent understanding of how HEMTs can be engineered to meet 
diverse functional requirements within high-frequency systems.
