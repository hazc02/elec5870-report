\chapter{Introduction and Technical Background}
\label{ch:introduction}

\section{Motivation and Context}
This project investigates the feasibility of achieving enhancement-mode (e-mode)
behaviour in GaAs-based pseudomorphic high-electron-mobility transistors
(pHEMTs) using controlled gate recessing. While CMOS dominates general-purpose
logic, III--V devices retain strong relevance in RF, mixed-signal, and
radiation-hard domains owing to high carrier mobility and favourable
transport. The central question is whether positive-threshold devices can be
realised on GaAs without an explicit etch-stop layer---opening a pathway to
logic topologies that leverage compound-semiconductor performance.

\section{Device Physics Background}
A pHEMT typically comprises a GaAs channel and an AlGaAs barrier that forms a
two-dimensional electron gas (2DEG) at the heterointerface. A Schottky gate
electrostatically modulates the channel. Depletion-mode (d-mode) operation
arises when the channel is conductive at zero gate bias (negative threshold),
whereas e-mode devices conduct only for positive gate bias. Gate recessing
reduces gate-to-channel separation and can shift the threshold voltage, but is
highly sensitive to surface states and processing chemistry.

\begin{figure}[H]
  \centering
  % Placeholder cross-section schematic
  \fbox{\rule{0pt}{40mm}\rule{0.9\linewidth}{0pt}}
  \caption{Simplified pHEMT cross-section showing gate, barrier, channel, and
  recess region. Replace with final schematic from your CAD/diagram tool.}
  \label{fig:phemt-cross-section}
\end{figure}

\section{Relevance to Logic}
GaAs logic families (e.g., DCFL and e/d-load logic) benefit from devices with a
positive threshold voltage. The project aims to fabricate and characterise
recessed-gate pHEMTs that enable such topologies, providing an empirical link
between recess depth and threshold voltage.
