\chapter{Introduction}

\section{Context and Motivation}

Modern communication, sensing and microwave systems place increasing demands on
the semiconductor devices that form their front-end signal paths. Higher
operating frequencies, lower permissible noise figures and stricter linearity
requirements all intensify the performance constraints on transistor
technologies. Although silicon-based RF platforms have progressed considerably, they do not always
deliver the mobility, noise performance or gain required in
performance-driven microwave applications.

III--V high-electron-mobility transistors (HEMTs) continue to address this gap.
Their favourable material properties, including high electron mobility and low
intrinsic noise, support predictable and efficient operation at microwave and
millimetre-wave frequencies \cite{ali1991}. As a result, HEMTs remain essential 
in systems such as satellite communication payloads, radar receivers, 
millimetre-wave backhaul links and scientific instrumentation, where stringent 
noise and frequency requirements cannot be met by silicon alone.

\section{Relevance of GaAs/AlGaAs HEMTs to This Project}

The GaAs/AlGaAs HEMT is one of the most established III--V transistor
architectures. Its behaviour is strongly influenced by the quality of the
heterostructure and the precision of the fabrication sequence, making it
well-suited to a project that aims to evaluate how material properties and
process decisions shape device performance. In this work, GaAs/AlGaAs HEMTs are
fabricated within the Leeds Nanotechnology Cleanroom to gain practical
experience with device definition, assess process reproducibility and examine
the electrical characteristics that emerge from the epitaxial structure.

A central motivation is the transition from the depletion-mode operation
inherent to the supplied wafer structure toward enhancement-mode behaviour
through controlled gate recessing. Achieving enhancement mode requires reducing
the effective gate-to-channel separation while maintaining interface quality.
This places strict demands on etch controllability, surface integrity and
uniformity across devices. Establishing the relationship between recess depth
and threshold voltage therefore represents a key technical objective.

\section{Project Motivation and Aims}

The first stage of the project focuses on fabricating and characterising
baseline depletion-mode GaAs/AlGaAs HEMTs. These devices provide the necessary
framework for assessing the heterostructure, validating the fabrication
workflow and establishing the electrical measurement setup. A major aim is to
understand how threshold voltage, transconductance and gate leakage relate to
the epitaxial design and the process conditions used to form the contacts and
gate.

Building on this foundation, the project then seeks to develop a controlled
gate-recess process capable of shifting the threshold voltage into the
enhancement regime. This requires a systematic study of etch depth, surface
behaviour and process reproducibility. A longer-term motivation is the eventual
construction of simple logic elements using enhancement-mode GaAs-based
devices, since enhancement-mode operation provides defined switching thresholds
and more flexible circuit behaviour appropriate for digital logic.