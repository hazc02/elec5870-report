\chapter{Introduction}

Modern communication, sensing and microwave systems impose increasingly 
demanding requirements on the semiconductor devices that form their 
front-end signal paths. Higher operating frequencies, tighter noise 
constraints and strict linearity targets continue to push transistor 
technologies beyond the capabilities of many silicon-based platforms. 
Although silicon processes have made notable progress in recent years, 
they do not always provide the carrier mobility, noise performance or 
high-frequency gain required in performance-critical microwave and 
millimetre-wave applications.

III--V High--Electron--Mobility Transistors (HEMTs) play a central role in 
addressing these challenges. Their heterostructure--based design creates a 
highly conductive channel at the interface between two semiconductor layers 
with different bandgaps. At this interface a two--dimensional electron gas (2DEG) 
forms, supporting exceptionally high carrier mobility. The resulting strong 
transconductance, stable high--frequency behaviour and low--noise performance 
have firmly established HEMTs as core devices throughout microwave and 
millimetre--wave circuit design.

Research activity within the field has increasingly shifted toward 
gallium nitride (GaN) HEMTs, driven by the material properties enabled 
by its wide bandgap. GaN supports high breakdown voltages, strong 
power--handling capability and robust operation under large electric 
fields, making it the preferred choice for many high--power and 
high--linearity applications \todobib. Despite this growing dominance, 
gallium arsenide (GaAs) HEMTs remain one of the most mature and well 
characterised III--V transistor platforms. Their stability, extensive 
documentation and long industrial history continue to make them an 
invaluable system for studying the relationships between heterostructure 
design, device architecture and electrical behaviour.

For the purposes of this project, GaAs presents clear practical and 
technical advantages. The Leeds Nanotechnology Cleanroom maintains 
established, reliable processes for GaAs/AlGaAs heterostructures, and 
suitable wafers are readily available with reproducible epitaxial 
quality \cite{leedscleanroom}. This provides a controlled environment in 
which HEMT structures can be fabricated and analysed systematically. The 
work undertaken here aims to investigate how variations in device 
structure influence key electrical characteristics, including threshold 
voltage, transconductance, leakage behaviour and the overall conduction 
profile of the device. The intention is to determine how specific 
structural changes translate into measurable shifts in device 
performance, building a clear link between fabrication choices and the 
resulting electrical behaviour.

