\chapter{Literature Review}
\label{ch:litreview}

\section{Enhancement-Mode Methods}
Reported routes to e-mode behaviour in AlGaAs/GaAs structures include double
recess geometries and metallurgical gate approaches (e.g., Pt sinking). Double
recess structures can yield uniform, positive thresholds with careful control
of barrier thinning, while gate sinking modifies the effective Schottky
barrier. Oxide/dielectric engineering also appears in the literature.

Representative works include \cite{hsieh2004,chu2007}. Update and expand this
section with additional sources as your reading develops.

\section{Comparative Analysis}
\begin{table}[H]
  \centering
  \caption{Comparison of enhancement-mode methods and practical considerations.}
  \label{tab:method-comparison}
  \begin{tabular}{@{}lllll@{}}
    \toprule
    Method & Wafer needs & Pros & Cons & Local feasibility \\
    \midrule
    Double recess & Etch control & Uniform V$_{\mathrm{th}}$ & Process complexity & High \\
    Gate sinking & Specific metals & Simple flow & Metallurgical risk & Medium \\
    Dielectric eng. & Deposition & Interface tuning & Additional tooling & Low--Med \\
    \bottomrule
  \end{tabular}
\end{table}
A template for comparing techniques, wafer requirements, and local feasibility
is provided in Table~\ref{tab:method-comparison}.
\section{Identified Gap}
There is limited reporting on controlled recessing without an etch-stop using a
process flow similar to that available at Leeds. This work targets an empirical
map of recess depth versus threshold voltage under those constraints.
