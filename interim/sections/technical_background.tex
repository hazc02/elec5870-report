\chapter{Technical Background}
\label{chap:technical_background}

\section{Overview of GaAs/AlGaAs HEMTs}

In the GaAs/AlGaAs HEMTs used in this project, the interface between the 
GaAs channel layer and the overlying AlGaAs barrier produces a 
discontinuity in the conduction band because the two materials have 
different bandgaps. This offset creates a potential well on the GaAs side 
of the junction, as demonstrated in \ref{fig:band_bending}. Electrons supplied by silicon donors in the modulation--doped 
AlGaAs layer transfer into this well and accumulate to form the 
2DEG that serves as the conducting channel. This 2DEG forming at the interface is the 
defining feature of the HEMT structure \cite{liu1999}.

% figures/fig-band-bending.tex
\begin{figure}[H]
  \centering
  \vspace{-0.75cm} % lift drawing to reduce apparent whitespace
  \begin{tikzpicture}[x=1.2cm,y=0.7cm,>=Stealth,font=\footnotesize]

    % Axes labels
    \node[anchor=north] at (2.2,0) {AlGaAs};
    \node[anchor=north] at (5.7,0) {GaAs};

    % Fermi level reference
    \draw[red, dashed] (0.2,3) -- (6.8,3);
    \node[anchor=east] at (0.1,3) {$E_F$};

    % Conduction band (bend into a narrow well then flatten in GaAs)
    \draw
      (0.2,3.3) -- (3.5,3.3)
      .. controls (4.5,3.3) and (4.5,6) .. (4.5,2.5)
      .. controls (5,3.8) and (5.3,3.8) .. (5.4,3.8)
      -- (6.6,3.8);

    % Valence band (S-bend at the interface then flat)
    \draw
      (0.2,0) -- (3.8,0)
      .. controls (4.5,0) and (4.5,0.6) .. (4.5,1)
      .. controls (4.5,1.5) and (5,1.5) .. (5.6,1.5)
      -- (6.6,1.5);

    % Energy labels on left
    \node[anchor=east] at (0.1,3.5) {$E_C$};
    \node[anchor=east] at (0.1,0) {$E_V$};

    % Highlight a small triangular 2DEG region in the well
    \filldraw[blue!70] (4.5,2.5) -- (4.5,3) -- (4.7,3) -- cycle;

    % 2DEG indication at the well minimum
    \draw[->,blue,thick] (6.5,2.5) -- (4.65,2.7);
    \node[blue,anchor=south west] at (6.4,2.25) {2DEG};

  \end{tikzpicture}
  \vspace{5mm}
  \caption{Band Structure Diagram of the AlGaAs/GaAs interface demonstrating the resulting Two-Dimensional Electron Gas (2DEG) formed in a Potential Well.}
  \label{fig:band_bending}
\end{figure}


\section{The GaAs/AlGaAs Heterostructure}

The HEMTs used in this project are based on a conventional GaAs/AlGaAs
heterostructure in which a sequence of epitaxial layers is engineered to
create and control a two--dimensional electron gas (2DEG). A schematic
cross--section of the structure is shown in Figure~\ref{fig:layer_structure},
highlighting the cap, donor, spacer and channel layers that together
define the electrical behaviour of the device.

\vspace{2mm}
% ------------------------------------------------------------
% File: figures/fig-layer-structure.tex
% GaAs/AlGaAs HEMT Layer Structure (Cross-Section)
% ------------------------------------------------------------

\begin{figure}[htbp]
    \vspace{2mm}
    \centering
    \begin{tikzpicture}[x=1.05cm, y=0.48cm, >=Stealth, font=\footnotesize]

        % Layer thicknesses
        \def\w{8}
        \def\hcap{1}
        \def\hdonor{2}
        \def\hspacer{1}
        \def\hchan{1.5}
        \def\hsub{1.5}

        % Coordinates
        \coordinate (subB) at (0,0);
        \coordinate (subT) at (0,\hsub);
        \coordinate (chanT) at (0,\hsub+\hchan);
        \coordinate (spacerT) at (0,\hsub+\hchan+\hspacer);
        \coordinate (donorT) at (0,\hsub+\hchan+\hspacer+\hdonor);
        \coordinate (capT) at (0,\hsub+\hchan+\hspacer+\hdonor+\hcap);

        % Bounding box
        \path[use as bounding box]
        (-2.4,0) rectangle
        (\w+2.4,\hsub+\hchan+\hspacer+\hdonor+\hcap+1);

        % Substrate
        \fill[gray!15] (subB) rectangle ++(\w,\hsub);
        \draw (subB) rectangle ++(\w,\hsub);
        \node at (\w/2,\hsub/2)
            {Semi-insulating GaAs substrate};

        % Channel
        \fill[blue!8] (subT) rectangle ++(\w,\hchan);
        \draw (subT) rectangle ++(\w,\hchan);
        \node at (\w/2,\hsub+\hchan/2)
            {Undoped GaAs channel};

        % Spacer
        \fill[green!8] (chanT) rectangle ++(\w,\hspacer);
        \draw (chanT) rectangle ++(\w,\hspacer);
        \node at (\w/2,\hsub+\hchan+\hspacer/2)
            {Undoped AlGaAs spacer};
        \node at (-0.6,\hsub+\hchan+\hspacer/2) {$\approx$ 5nm};

        % Donor Layer
        \fill[yellow!20] (spacerT) rectangle ++(\w,\hdonor);
        \draw (spacerT) rectangle ++(\w,\hdonor);
        \node at (\w/2,\hsub+\hchan+\hspacer+\hdonor/2)
            {Si-doped AlGaAs donor layer};
        \node at (-0.5,\hsub+\hchan+\hspacer+\hdonor/2) {40nm};

        % Cap Layer
        \fill[red!15] (donorT) rectangle ++(\w,\hcap);
        \draw (donorT) rectangle ++(\w,\hcap);
        \node at (\w/2,\hsub+\hchan+\hspacer+\hdonor+\hcap/2)
            {n$^{+}$ GaAs cap layer};
        \node at (-0.5,\hsub+\hchan+\hspacer+\hdonor+\hcap/2) {10nm};


        % 2DEG line
        \draw[line width=1pt, blue]
            (chanT) -- ++(\w,0);
        \node[blue, anchor=west] at (\w+0.2,\hsub+\hchan)
            {2DEG at interface};

        % Contacts
        \fill[black!30]
            (0.7,\hsub+\hchan+\hspacer+\hdonor+\hcap)
            rectangle ++(0.7,0.55);
        \node[anchor=south] at
            (1.05,\hsub+\hchan+\hspacer+\hdonor+\hcap+0.55)
            {Source};

        \fill[black!30]
            (\w-1.4,\hsub+\hchan+\hspacer+\hdonor+\hcap)
            rectangle ++(0.7,0.55);
        \node[anchor=south] at
            (\w-1.05,\hsub+\hchan+\hspacer+\hdonor+\hcap+0.55)
            {Drain};

        % Gate
        \fill[black]
            ({\w/2-0.58},\hsub+\hchan+\hspacer+\hdonor+\hcap)
            rectangle ++(1.16,0.75);
        \node[anchor=south] at
            (\w/2,\hsub+\hchan+\hspacer+\hdonor+\hcap+0.75)
            {Gate};

    \end{tikzpicture}

    \vspace{5mm}

    \caption{Cross-section of the GaAs/AlGaAs HEMT used in this project (not to scale).}
    \label{fig:layer_structure}
\end{figure}

\vspace{5mm}

At the top of the structure lies a thin, highly doped GaAs cap layer.
This cap provides a stable surface for forming the Schottky gate contact
and prevents oxidation of the underlying AlGaAs. Its high conductivity
also assists charge access to the regions adjoining the source and drain
contacts.

\vspace{2mm}

Beneath the cap is the Si--doped AlGaAs donor layer. Silicon donors in
AlGaAs supply electrons that preferentially transfer into the GaAs
channel due to the conduction--band discontinuity at the heterointerface.
This arrangement is known as modulation doping, as the dopants are placed
in the barrier rather than in the channel. The donor concentration and
its distance from the interface set the available carrier population for
the 2DEG.

\vspace{2mm}

A thin undoped AlGaAs spacer layer separates the donor layer from the
GaAs channel. Although only a few nanometres thick, it plays a crucial
role in determining device performance. By physically separating the
2DEG from the ionised silicon donors, the spacer suppresses Coulomb
scattering and allows carriers to move through the channel with
significantly higher mobility. This benefit becomes particularly
pronounced at cryogenic temperatures, where phonon scattering is reduced,
 and impurity scattering would otherwise dominate. The spacer thickness
therefore influences both the electrostatic coupling to the gate and the
mobility of carriers within the 2DEG.

\vspace{2mm}

The GaAs channel beneath the spacer hosts the 2DEG. Electrons accumulate
at the AlGaAs/GaAs interface, where the conduction--band offset creates
a quantum--well--like potential that confines carriers to a narrow region
within the GaAs. This interfacial channel is responsible for the
high-mobility transport behaviour characteristic of GaAs/AlGaAs HEMTs.

\vspace{2mm}

Together, these layers define the depth, density and confinement of the
2DEG, providing the structural basis for the electrical behaviour
analysed later in this report.

\section{HEMT Operating Principles}

The electrical behaviour of the GaAs/AlGaAs HEMTs used in this project is
governed by the interaction between the Schottky gate contact and the
two--dimensional electron gas (2DEG) formed within the heterostructure.
When a metal is deposited on the GaAs surface, a Schottky barrier is
formed, giving rise to a depletion region beneath the gate. Applying a
gate voltage modulates this depletion region and enables electrostatic
control of the 2DEG at the AlGaAs/GaAs interface.

\vspace{2mm}

A negative gate voltage widens the depletion region and reduces the
electron density within the underlying 2DEG. As the voltage becomes more
negative, the carrier concentration continues to fall until the 2DEG is
fully depleted; the corresponding gate voltage defines the threshold
voltage of the device.

\vspace{2mm}

When a drain voltage is applied, electrons flow laterally through the
2DEG from source to drain. At low drain voltages the channel remains
uniformly accumulated, resulting in a linear relationship between
drain current and drain voltage. As the drain voltage increases, the
channel near the drain becomes progressively depleted, leading to
pinch--off. Beyond this point the drain current saturates, with the
saturation level determined primarily by the gate voltage.

\vspace{2mm}

The sensitivity of the drain current to changes in gate voltage is
expressed through the transconductance, $g_{m} = \partial I_{D} /
\partial V_{G}$. High values of $g_{m}$ arise from strong electrostatic
coupling between the gate and the 2DEG, together with the high intrinsic
mobility of carriers within the channel.

\vspace{2mm}

The Schottky gate introduces a finite leakage current, particularly at
large negative gate biases where thermionic emission and tunnelling
processes become more significant. The magnitude of this leakage depends
on the quality of the gate interface, the barrier height and the
condition of the semiconductor surface.

\vspace{2mm}

Together, these electrostatic and transport processes describe how a
GaAs/AlGaAs HEMT functions as a field--effect transistor and provide the
basis for the electrical characterisation presented later in this report.