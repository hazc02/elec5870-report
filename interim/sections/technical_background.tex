\chapter{Technical Background}
\label{chap:technical_background}

\section{Overview of GaAs/AlGaAs High Electron Mobility Transistors}

High electron mobility transistors (HEMTs) utilise a semiconductor
heterojunction to create a conducting channel with markedly improved electron
transport compared with uniformly doped structures. In the GaAs/AlGaAs system
employed in this project, the difference in bandgap between the two materials
establishes a potential well at their interface. Electrons supplied from the
wider-bandgap AlGaAs layer transfer into the narrower-bandgap GaAs, forming a
highly mobile sheet of charge known as a two-dimensional electron gas (2DEG).
This confined channel supports rapid carrier transport, enabling high
transconductance and excellent high-frequency behaviour, as described in
standard texts on III--V devices\cite{liu1999,ali1991}.

The devices studied and fabricated in this project are based on this
GaAs/AlGaAs heterostructure. The epitaxial stack comprises an undoped GaAs
channel, an undoped AlGaAs spacer separating the channel from ionised donors,
and a silicon-doped AlGaAs barrier which supplies electrons. Above these
layers sits a highly doped GaAs cap that enables practical device fabrication,
providing a clean, conductive surface for ohmic metallisation and gate
definition. A schematic cross-section is provided in
Figure~\ref{fig:hemt_xsec}, and the subsequent sections expand on the
structure's electronic function and its relevance to device operation.

\begin{figure*}[t]
  \centering
  \begin{tikzpicture}[x=1.08cm,y=0.5cm,>=Stealth,font=\footnotesize]

    % Layer thicknesses (in arbitrary units)
    \def\w{6}      % wafer width
    \def\hcap{0.75}   % n+ GaAs cap
    \def\hdonor{2} % Si-doped AlGaAs
    \def\hspacer{0.75}% undoped AlGaAs spacer
    \def\hchan{2}  % undoped GaAs channel
    \def\hsub{1.5}   % semi-insulating substrate

    % y positions (bottom to top)
    \coordinate (subB) at (0,0);
    \coordinate (subT) at (0,\hsub);
    \coordinate (chanT) at (0,\hsub+\hchan);
    \coordinate (spacerT) at (0,\hsub+\hchan+\hspacer);
    \coordinate (donorT) at (0,\hsub+\hchan+\hspacer+\hdonor);
    \coordinate (capT) at (0,\hsub+\hchan+\hspacer+\hdonor+\hcap);

    % Symmetric bounding box so side labels do not shift centering
    \path[use as bounding box]
      (-2.5,0) rectangle (\w+2.5,\hsub+\hchan+\hspacer+\hdonor+\hcap+1);

    % Semi-insulating GaAs substrate
    \fill[gray!15]
      (subB) rectangle ++(\w,\hsub);
    \draw (subB) rectangle ++(\w,\hsub);
    \node[white,opacity=0] at (\w/2,\hsub/2) {};
    \node[align=center] at (\w/2,\hsub/2)
      {Semi-insulating GaAs substrate};

    % Undoped GaAs channel
    \fill[blue!8]
      (subT) rectangle ++(\w,\hchan);
    \draw (subT) rectangle ++(\w,\hchan);
    \node[align=center] at (\w/2,\hsub+\hchan/2)
      {Undoped GaAs channel};

    % Undoped AlGaAs spacer
    \fill[green!8]
      (chanT) rectangle ++(\w,\hspacer);
    \draw (chanT) rectangle ++(\w,\hspacer);
    \node[align=center] at (\w/2,\hsub+\hchan+\hspacer/2)
      {Undoped AlGaAs spacer};

    % Si-doped AlGaAs donor layer
    \fill[yellow!20]
      (spacerT) rectangle ++(\w,\hdonor);
    \draw (spacerT) rectangle ++(\w,\hdonor);
    \node[align=center] at (\w/2,\hsub+\hchan+\hspacer+\hdonor/2)
      {Si-doped AlGaAs donor layer};

    % n+ GaAs cap
    \fill[red!15]
      (donorT) rectangle ++(\w,\hcap);
    \draw (donorT) rectangle ++(\w,\hcap);
    \node[align=center] at (\w/2,\hsub+\hchan+\hspacer+\hdonor+\hcap/2)
      {n$^{+}$ GaAs cap layer};

    % 2DEG at the GaAs/AlGaAs interface
    \draw[line width=1pt,blue]
      (chanT) -- ++(\w,0);
    \draw[line width=1pt,blue]
      (\w,\hsub+\hchan) -- (\w,\hsub+\hchan);
    \node[blue,align=left,anchor=west] at (\w+0.2,\hsub+\hchan)
      {2DEG at GaAs/AlGaAs interface};

    % Source and drain ohmic metal (simplified)
    \fill[black!30]
      (0.6,\hsub+\hchan+\hspacer+\hdonor+\hcap)
      rectangle ++(0.7,0.6);
    \fill[black!30]
      (\w-1.3,\hsub+\hchan+\hspacer+\hdonor+\hcap)
      rectangle ++(0.7,0.6);
    \node[anchor=south] at
      (0.95,\hsub+\hchan+\hspacer+\hdonor+\hcap+0.6)
      {Source};
    \node[anchor=south] at
      (\w-0.95,\hsub+\hchan+\hspacer+\hdonor+\hcap+0.6)
      {Drain};

    % Gate metal (Schottky)
    \fill[black]
      ({\w/2-0.6},\hsub+\hchan+\hspacer+\hdonor+\hcap)
      rectangle ++(1.2,0.8);
    \node[anchor=south] at
      (\w/2,\hsub+\hchan+\hspacer+\hdonor+\hcap+0.8)
      {Gate};

  \end{tikzpicture}
  \caption[GaAs/AlGaAs HEMT cross-section]%
  {GaAs/AlGaAs HEMT cross-section (not to scale). Labels are inset to reduce
  clutter; the two-dimensional electron gas (2DEG) forms at the GaAs/AlGaAs
  interface.}
  \label{fig:hemt_xsec}
\end{figure*}

\section{The GaAs/AlGaAs Heterostructure}
\label{sec:heterostructure}

The electronic behaviour of the HEMT derives from its carefully engineered
epitaxial heterostructure. The arrangement of the GaAs and AlGaAs layers
dictates how electrons are distributed, confined and subsequently modulated by
the gate. Only three layers form the active electronic structure responsible
for the formation of the high-mobility channel: the undoped GaAs channel, the
undoped AlGaAs spacer, and the Si-doped AlGaAs donor layer. These layers
determine the depth and population of the two-dimensional electron gas and
therefore underpin the device characteristics measured later in the report.

\subsection{Undoped GaAs Channel}

The lowest active layer in the structure is an undoped GaAs region, which
serves as the host material for the two-dimensional electron gas. Its absence
of intentional doping ensures that electrons do not encounter ionised
impurities within the conduction path; this greatly reduces scattering. The
uniformity and crystalline quality of this GaAs layer are essential, as the
electronic properties of the device depend sensitively on the interface
quality at the GaAs/AlGaAs boundary. The thickness of the channel is chosen
so that electrons remain tightly confined near the interface without
introducing strain or dislocations.

\subsection{Undoped AlGaAs Spacer}

Directly above the GaAs channel lies a thin undoped AlGaAs spacer. Although
only a few nanometres thick, its role is critical. The spacer physically
separates the conduction channel from the doped barrier layer above it. This
separation prevents ionised donor atoms from residing within the channel
region, significantly suppressing Coulomb scattering. At the same time, the
spacer remains thin enough to maintain strong electrostatic coupling between
the donor layer and the channel. The balance between these requirements is
central to achieving both high sheet density and high mobility.

\subsection{Si-Doped AlGaAs Donor Layer}

The Si-doped AlGaAs layer provides the electrons that populate the channel.
Its larger bandgap, compared with GaAs, places its conduction band edge at a
higher energy. Electrons therefore diffuse into the GaAs channel until
equilibrium is reached, leaving behind positively charged ionised donors in
the AlGaAs. The aluminium composition determines the magnitude of the
conduction band offset, while the doping concentration sets the density of
available electrons. Together, these parameters determine the depth of the
potential well at the interface and the resulting sheet electron density in
the two-dimensional electron gas.

\subsection{Summary of Heterostructure Behaviour}

These three layers act together to create a structure in which electrons are
supplied by the AlGaAs donor layer, confined within the GaAs channel, and
shielded from ionised impurities by the spacer. The resulting electronic
environment enables high carrier mobility, strong gate control and the
characteristic performance advantages associated with GaAs/AlGaAs HEMTs. This
foundation supports the device operation described in the following sections
and informs many of the fabrication decisions discussed later in the report.

\section{Formation and Properties of the Two-Dimensional Electron Gas}
\label{sec:2deg}

Electron transfer from the doped AlGaAs barrier into the undoped GaAs results
in the formation of a two-dimensional electron gas at their interface. This
occurs because the conduction band in GaAs lies at a lower energy than in
AlGaAs; a potential well is therefore formed at the heterojunction. Electrons
spill over from the AlGaAs into this well and accumulate in a very thin region
close to the interface. The channel thickness is much smaller than in a
conventionally doped epilayer, which enhances carrier mobility by reducing the
probability of scattering within the conducting layer.

The absence of intentional doping in the GaAs channel eliminates ionised
impurity scattering, which is typically one of the dominant mechanisms
limiting mobility in standard GaAs transistors\cite{liu1999,ali1991}. Instead,
carriers in the two-dimensional electron gas experience primarily phonon
scattering at room temperature and, to a lesser extent, interface roughness.
These mechanisms still influence mobility, but their combined effect is
significantly weaker than the impurity-driven scattering present in a
uniformly doped channel. The resulting electron sheet density is typically
high, since the doped AlGaAs layer supplies a substantial reservoir of
electrons. This high density, combined with the improved mobility, produces a
channel with strong conductivity and excellent transconductance potential.

The characteristics of the two-dimensional electron gas therefore underpin the
performance of GaAs/AlGaAs HEMTs and explain their widespread use in
high-frequency and low-noise applications. In the context of this project,
they also provide the link between the epitaxial design and the measured drain
current, threshold and transconductance obtained from the fabricated devices.

\section{Practical Device Layers: n$^{+}$ GaAs Cap and Schottky Gate Interface}
\label{sec:practical_layers}

Above the active heterostructure sits a highly doped GaAs cap layer. This cap
is central to achieving reliable ohmic contacts, as its heavy doping allows
metal alloys such as AuGe/Ni/Au to penetrate and form a low-resistance
connection to the underlying two-dimensional electron gas during the annealing
process. The cap also defines the surface on which lithography is performed,
providing a uniform, conductive layer that facilitates consistent gate
formation across the wafer.

The gate electrode forms a Schottky contact with the underlying material. When
a gate metal such as Ti/Au is deposited onto the GaAs or AlGaAs surface, a
rectifying junction is created. This junction is responsible for controlling
the depletion region within the AlGaAs barrier; by adjusting the gate--source
voltage, the electric field in the barrier is modified, which in turn changes
the electron density at the heterointerface. The Schottky barrier therefore
replaces the oxide--semiconductor interface used in MOSFETs, and its behaviour
directly determines the degree of control the gate has over the channel.

These layers do not contribute to the formation of the two-dimensional electron
gas itself, but they make the device physically realisable. Their thickness,
doping level and processing quality influence the final device characteristics,
particularly contact resistance and the efficiency with which the gate can
modulate the channel.

\section{Operation of the GaAs/AlGaAs HEMT}
\label{sec:hemt_operation}

The operation of the GaAs/AlGaAs HEMT relies on the ability of the gate
electrode to modulate the electron density within the two-dimensional electron
gas. By applying a voltage to the Schottky gate, the depletion region in the
AlGaAs barrier can be expanded or contracted. When the gate potential becomes
sufficiently negative relative to the source, this depletion region approaches
the channel and begins to reduce the available sheet charge. As the electron
density beneath the gate falls, the drain current correspondingly decreases.

In the output characteristics, the device behaves resistively at low
drain--source voltages, where the two-dimensional electron gas is uniformly
populated along the channel. As the drain voltage increases, the potential
near the drain end of the gate reduces the local electron density, leading to
pinch-off and the onset of current saturation. This transition marks the point
at which an increase in electric field no longer produces a proportional
increase in carrier velocity, as velocity saturation in GaAs becomes
significant.

The gate voltage also strongly influences the transconductance, as the
relationship between the applied bias and the resulting change in sheet charge
is highly sensitive. Changes in electron density translate directly into
changes in conductivity, and therefore into changes in drain current. The high
mobility of the two-dimensional electron gas and the relatively small distance
between the gate and the conducting channel enable strong electrostatic
control, giving the HEMT its characteristic high transconductance and
excellent amplification capability.

\section{Summary}
\label{sec:tb_summary}

The GaAs/AlGaAs HEMT achieves its performance through a carefully engineered
heterostructure in which electrons donated from a doped AlGaAs layer populate
an undoped GaAs channel, forming a high-mobility two-dimensional electron gas.
A thin AlGaAs spacer separates the channel from the donors, suppressing
impurity scattering while maintaining strong electrostatic coupling. A highly
doped GaAs cap and a Schottky gate interface make the structure practically
usable, enabling low-resistance ohmic contacts and effective gate control of
the channel.

Together, these features yield a transistor with high carrier mobility, strong
gate modulation and favourable current--voltage characteristics. This
technical background provides the basis for understanding the fabrication
sequence and electrical measurements presented in the subsequent chapters of
this report.
