\chapter{Technical Background}
\label{chap:technical_background}

\section{Overview of GaAs/AlGaAs HEMTs}

High-electron-mobility transistors employ a semiconductor heterojunction to
create a confined, low-scattering electron channel. In the GaAs/AlGaAs material
system used in this project, a discontinuity in the conduction band establishes
a potential well at the interface. Electrons supplied by the doped AlGaAs layer
accumulate within this well and form a two-dimensional electron gas (2DEG).
Because the GaAs channel is undoped, the absence of ionised impurities suppresses
Coulomb scattering, leading to high electron mobility and predictable
high-frequency behaviour \cite{liu1999}.

The epitaxial structure used in this project is a standard, lattice-matched
GaAs/AlGaAs heterostructure. Unlike pseudomorphic devices that rely on strained
InGaAs channels, this system achieves confinement purely through band offsets,
ensuring stable material behaviour without the additional complexity of strain
engineering.

% figures/fig-band-bending.tex
\begin{figure}[H]
  \centering
  \vspace{-0.75cm} % lift drawing to reduce apparent whitespace
  \begin{tikzpicture}[x=1.2cm,y=0.7cm,>=Stealth,font=\footnotesize]

    % Axes labels
    \node[anchor=north] at (2.2,0) {AlGaAs};
    \node[anchor=north] at (5.7,0) {GaAs};

    % Fermi level reference
    \draw[red, dashed] (0.2,3) -- (6.8,3);
    \node[anchor=east] at (0.1,3) {$E_F$};

    % Conduction band (bend into a narrow well then flatten in GaAs)
    \draw
      (0.2,3.3) -- (3.5,3.3)
      .. controls (4.5,3.3) and (4.5,6) .. (4.5,2.5)
      .. controls (5,3.8) and (5.3,3.8) .. (5.4,3.8)
      -- (6.6,3.8);

    % Valence band (S-bend at the interface then flat)
    \draw
      (0.2,0) -- (3.8,0)
      .. controls (4.5,0) and (4.5,0.6) .. (4.5,1)
      .. controls (4.5,1.5) and (5,1.5) .. (5.6,1.5)
      -- (6.6,1.5);

    % Energy labels on left
    \node[anchor=east] at (0.1,3.5) {$E_C$};
    \node[anchor=east] at (0.1,0) {$E_V$};

    % Highlight a small triangular 2DEG region in the well
    \filldraw[blue!70] (4.5,2.5) -- (4.5,3) -- (4.7,3) -- cycle;

    % 2DEG indication at the well minimum
    \draw[->,blue,thick] (6.5,2.5) -- (4.65,2.7);
    \node[blue,anchor=south west] at (6.4,2.25) {2DEG};

  \end{tikzpicture}
  \vspace{5mm}
  \caption{Band Structure Diagram of the AlGaAs/GaAs interface demonstrating the resulting Two-Dimensional Electron Gas (2DEG) formed in a Potential Well.}
  \label{fig:band_bending}
\end{figure}


\section{The GaAs/AlGaAs Heterostructure}

The heterostructure consists of three active layers.  
The undoped GaAs channel forms the conduction path for the 2DEG. Its crystalline
uniformity and lack of intentional impurities suppress scattering and enable
high mobility.  
Above the channel lies the undoped AlGaAs spacer, which physically separates the
electrons from the ionised donors in the doped barrier. Its thickness
determines the balance between mobility and electrostatic coupling.  
The Si-doped AlGaAs donor layer then supplies electrons and establishes the
conduction band offset needed to form the potential well at the interface.

Together, these layers determine the electron density, mobility and confinement
properties that underpin HEMT behaviour.

\section{Formation and Properties of the Two-Dimensional Electron Gas}

Electrons diffuse from the doped AlGaAs barrier into the undoped GaAs channel
until equilibrium is established. The conduction band offset forms a narrow,
triangular potential well that confines the carriers to the interface. Because
the channel contains no ionised donors, scattering from charged impurities is
largely eliminated \cite{ali1991}. Mobility is therefore limited primarily by
phonon interactions and interface roughness, both of which are less severe
mechanisms.  
This combination of strong confinement and reduced scattering results in a
high-mobility electron sheet that supports strong transconductance and
predictable current–voltage behaviour.

\section{Practical Device Layers and Gate Structure}

Above the active heterostructure sits the heavily doped GaAs cap. Its high
doping level enables reliable formation of low-resistance ohmic contacts when
the AuGe/Ni/Au metallisation stack is alloyed, while also providing a uniform,
conductive surface for lithographic patterning.

The Ti/Au Schottky gate controls the depletion region within the AlGaAs
barrier. Adjusting the gate bias modifies the electric field beneath the gate
and therefore the electron density in the 2DEG. The effective
gate-to-channel separation plays a critical role in determining the threshold
voltage, transconductance and linearity of the device.  
Unlike MOSFETs, HEMTs do not employ an insulating oxide; control is achieved
purely through depletion of the channel.

% ------------------------------------------------------------
% File: figures/fig-layer-structure.tex
% GaAs/AlGaAs HEMT Layer Structure (Cross-Section)
% ------------------------------------------------------------

\begin{figure}[htbp]
    \vspace{2mm}
    \centering
    \begin{tikzpicture}[x=1.05cm, y=0.48cm, >=Stealth, font=\footnotesize]

        % Layer thicknesses
        \def\w{8}
        \def\hcap{1}
        \def\hdonor{2}
        \def\hspacer{1}
        \def\hchan{1.5}
        \def\hsub{1.5}

        % Coordinates
        \coordinate (subB) at (0,0);
        \coordinate (subT) at (0,\hsub);
        \coordinate (chanT) at (0,\hsub+\hchan);
        \coordinate (spacerT) at (0,\hsub+\hchan+\hspacer);
        \coordinate (donorT) at (0,\hsub+\hchan+\hspacer+\hdonor);
        \coordinate (capT) at (0,\hsub+\hchan+\hspacer+\hdonor+\hcap);

        % Bounding box
        \path[use as bounding box]
        (-2.4,0) rectangle
        (\w+2.4,\hsub+\hchan+\hspacer+\hdonor+\hcap+1);

        % Substrate
        \fill[gray!15] (subB) rectangle ++(\w,\hsub);
        \draw (subB) rectangle ++(\w,\hsub);
        \node at (\w/2,\hsub/2)
            {Semi-insulating GaAs substrate};

        % Channel
        \fill[blue!8] (subT) rectangle ++(\w,\hchan);
        \draw (subT) rectangle ++(\w,\hchan);
        \node at (\w/2,\hsub+\hchan/2)
            {Undoped GaAs channel};

        % Spacer
        \fill[green!8] (chanT) rectangle ++(\w,\hspacer);
        \draw (chanT) rectangle ++(\w,\hspacer);
        \node at (\w/2,\hsub+\hchan+\hspacer/2)
            {Undoped AlGaAs spacer};
        \node at (-0.6,\hsub+\hchan+\hspacer/2) {$\approx$ 5nm};

        % Donor Layer
        \fill[yellow!20] (spacerT) rectangle ++(\w,\hdonor);
        \draw (spacerT) rectangle ++(\w,\hdonor);
        \node at (\w/2,\hsub+\hchan+\hspacer+\hdonor/2)
            {Si-doped AlGaAs donor layer};
        \node at (-0.5,\hsub+\hchan+\hspacer+\hdonor/2) {40nm};

        % Cap Layer
        \fill[red!15] (donorT) rectangle ++(\w,\hcap);
        \draw (donorT) rectangle ++(\w,\hcap);
        \node at (\w/2,\hsub+\hchan+\hspacer+\hdonor+\hcap/2)
            {n$^{+}$ GaAs cap layer};
        \node at (-0.5,\hsub+\hchan+\hspacer+\hdonor+\hcap/2) {10nm};


        % 2DEG line
        \draw[line width=1pt, blue]
            (chanT) -- ++(\w,0);
        \node[blue, anchor=west] at (\w+0.2,\hsub+\hchan)
            {2DEG at interface};

        % Contacts
        \fill[black!30]
            (0.7,\hsub+\hchan+\hspacer+\hdonor+\hcap)
            rectangle ++(0.7,0.55);
        \node[anchor=south] at
            (1.05,\hsub+\hchan+\hspacer+\hdonor+\hcap+0.55)
            {Source};

        \fill[black!30]
            (\w-1.4,\hsub+\hchan+\hspacer+\hdonor+\hcap)
            rectangle ++(0.7,0.55);
        \node[anchor=south] at
            (\w-1.05,\hsub+\hchan+\hspacer+\hdonor+\hcap+0.55)
            {Drain};

        % Gate
        \fill[black]
            ({\w/2-0.58},\hsub+\hchan+\hspacer+\hdonor+\hcap)
            rectangle ++(1.16,0.75);
        \node[anchor=south] at
            (\w/2,\hsub+\hchan+\hspacer+\hdonor+\hcap+0.75)
            {Gate};

    \end{tikzpicture}

    \vspace{5mm}

    \caption{Cross-section of the GaAs/AlGaAs HEMT used in this project (not to scale).}
    \label{fig:layer_structure}
\end{figure}


\section{HEMT Operation}

HEMT operation is governed by the modulation of the 2DEG by the Schottky gate.
A negative gate bias expands the depletion region and reduces channel
conductivity. As the drain–source voltage increases, the electric field at the
drain side causes local depletion, eventually leading to pinch-off and current
saturation.  
The combination of high channel mobility and strong electrostatic coupling
between the gate and channel results in high transconductance, efficient
current modulation and favourable high-frequency behaviour.

\section{Summary}

The GaAs/AlGaAs HEMT derives its performance from engineered carrier
confinement, spatial separation from ionised donors and the electrostatic
control provided by the Schottky gate. These characteristics form the
technical foundation for the fabrication and characterisation work presented in
the subsequent chapters.