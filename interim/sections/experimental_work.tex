\chapter{Experimental Work to Date}
\label{ch:experimental}

\section{Depletion-Mode HEMT Fabrication and Testing}
Summarise the training run and provide a succinct fabrication flow. Include one
representative dataset (transfer and output characteristics) with captions that
explain the key features and extracted parameters (e.g., $V_{\mathrm{th}} \approx
-0.9\,\mathrm{V}$, $g_m \approx 3\,\mathrm{mS}$).

\begin{figure}[H]
  \centering
  % Placeholder for IV plot
  \fbox{\rule{0pt}{45mm}\rule{0.9\linewidth}{0pt}}
  \caption{Representative transfer characteristic for d-mode HEMT. Replace with
  plot from the Python pipeline.}
  \label{fig:dmode-transfer}
\end{figure}

\section{Python Data-Processing Pipeline}
Briefly describe the CSV parsing, plotting, and parameter-extraction stages.
Justify the pipeline for reproducibility and efficiency. Add a figure generated
by the tool once available.

\section{Enhancement-Mode Development (pHEMT)}
Describe etch-rate calibration (gas mixture, pressure, time) and summarise
profilometer measurements.

\begin{table}[H]
  \centering
  \caption{Example recess depth calibration (replace with measured data).}
  \label{tab:recess-depths}
  \begin{tabular}{@{}lll@{}}
    \toprule
    Etch time (s) & Depth (nm) & Notes \\
    \midrule
    10 & -- & calibration \\
    20 & -- & calibration \\
    30 & -- & device batch \\
    \bottomrule
  \end{tabular}
\end{table}
Table~\ref{tab:recess-depths} will be replaced with the measured relationship
between etch time and recess depth.
Include representative I--V and transfer curves indicating threshold shift, and
discuss trade-offs (surface roughness, leakage). Document any process
adjustments (post-etch clean, passivation).

\section{Discussion}
Compare recessed devices with the d-mode baseline, explain the physical origin
of threshold shifts, and comment on reproducibility and process sensitivity.
